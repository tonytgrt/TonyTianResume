%-------------------------
% Resume in Latex
% Author : Jake Gutierrez
% Based off of: https://github.com/sb2nov/resume
% License : MIT
%------------------------

\documentclass[letterpaper,11pt]{article}

\usepackage{latexsym}
\usepackage[empty]{fullpage}
\usepackage{titlesec}
\usepackage{marvosym}
\usepackage[usenames,dvipsnames]{color}
\usepackage{verbatim}
\usepackage{enumitem}
\usepackage[hidelinks]{hyperref}
\usepackage{fancyhdr}
\usepackage[english]{babel}
\usepackage{tabularx}
\usepackage[normalem]{ulem}
\input{glyphtounicode}



%----------FONT OPTIONS----------
% sans-serif
% \usepackage[sfdefault]{FiraSans}
% \usepackage[sfdefault]{roboto}
% \usepackage[sfdefault]{noto-sans}
% \usepackage[default]{sourcesanspro}

% serif
% \usepackage{CormorantGaramond}
% \usepackage{charter}


\pagestyle{fancy}
\fancyhf{} % clear all header and footer fields
\fancyfoot{}
\renewcommand{\headrulewidth}{0pt}
\renewcommand{\footrulewidth}{0pt}

% Adjust margins
\addtolength{\oddsidemargin}{-0.5in}
\addtolength{\evensidemargin}{-0.7in}
\addtolength{\textwidth}{1in}
\addtolength{\topmargin}{-0.75in}
\addtolength{\textheight}{1.0in}


\urlstyle{same}

%\raggedbottom
\raggedright
\setlength{\tabcolsep}{0in}

% Sections formatting
\titleformat{\section}{
  \vspace{-10pt}\scshape\raggedright\large
}{}{0em}{}[\color{black}\titlerule \vspace{-5pt}]

% Ensure that generate pdf is machine readable/ATS parsable
\pdfgentounicode=1

%-------------------------
% Custom commands
% Adjust underline depth to be closer to text
\renewcommand{\ULdepth}{2.5pt}

\newcommand{\resumeItem}[1]{
  \item\small{
    {#1 \vspace{-3pt}}
  }
}

\newcommand{\resumeSubheading}[4]{
  \vspace{-2pt}\item
    \begin{tabular*}{0.97\textwidth}[t]{l@{\extracolsep{\fill}}r}
      \textbf{#1} & #2 \\
      \textit{\small#3} & \textit{\small #4} \\
    \end{tabular*}\vspace{-7pt}
}

\newcommand{\resumeSubSubheading}[2]{
    \vspace{-7pt}\item
    \begin{tabular*}{0.97\textwidth}{l@{\extracolsep{\fill}}r}
      \textit{\small#1} & \textit{\small #2} \\
    \end{tabular*}\vspace{-7pt}
}

\newcommand{\resumeProjectHeading}[2]{
    \item
    \begin{tabular*}{0.97\textwidth}{l@{\extracolsep{\fill}}r}
      \small#1 & #2 \\
    \end{tabular*}\vspace{-7pt}
}

\newcommand{\resumeSubItem}[1]{\resumeItem{#1}\vspace{-4pt}}

\renewcommand\labelitemii{$\vcenter{\hbox{\tiny$\bullet$}}$}

\newcommand{\resumeSubHeadingListStart}{\begin{itemize}[leftmargin=0.15in, label={}]}
\newcommand{\resumeSubHeadingListEnd}{\end{itemize}}
\newcommand{\resumeItemListStart}{\begin{itemize}}
\newcommand{\resumeItemListEnd}{\end{itemize}\vspace{-5pt}}

%-------------------------------------------
%%%%%%  RESUME STARTS HERE  %%%%%%%%%%%%%%%%%%%%%%%%%%%%


\begin{document}

%----------HEADING----------
\begin{center}
    \textbf{\Huge \scshape Tony Yiding Tian} \\ \vspace{1pt}
    \small Philadelphia, PA $|$ (267)249-1202 $|$ \href{mailto:tonytg@seas.upenn.edu}{\underline{tonytg@seas.upenn.edu}} $|$
    \href{https://tonyxtian.com}{\underline{tonyxtian.com}}
\end{center}

%-----------EDUCATION-----------
\section{Education}
  \resumeSubHeadingListStart
    \resumeSubheading
      {University of Pennsylvania - School of Engineering and Applied Sciences}{Philadelphia, PA}
      {B.S.E. in Computer Engineering, Accelerated M.S.E. in Computer Graphics and Game Technology}{May 2027}
      \resumeItemListStart
        \resumeItem{GPA: 3.8 | Relevant Courses: GPU Programming, Advanced Rendering, Interactive Computer Graphics, Computer Animation, Operating Systems, Data Structures \& Algorithms, Computer Architecture}
      \resumeItemListEnd
  \resumeSubHeadingListEnd

  %-----------PROJECTS-----------
\section{Projects}
\resumeSubHeadingListStart

\resumeProjectHeading
      {\textbf{MatForge - Advanced Material Rendering System} $|$ \emph{Vulkan, Slang, C++}}{Nov 2025 -- Dec 2025}
      \resumeItemListStart
      \resumeItem{Architected production-quality GPU path tracer implementing four SIGGRAPH papers (2023-2024) in unified Vulkan ray tracing pipeline, contributing 2,400+ lines of C++ and Slang code for sampling and geometry systems}
      \resumeItem{Implemented Quad-Optimized Low-Discrepancy Sequences (QOLDS) from SIGGRAPH 2024, engineering base-3 Sobol' generator with Owen scrambling across 47 dimensions, achieving +2.57 dB PSNR and 44.7\% MSE reduction vs PCG sampling at 512 SPP with $<1\%$ performance overhead}
      \resumeItem{Developed RMIP (Rectangular MinMax Image Pyramid) intersection shader , enabling tessellation-free displacement mapping through hierarchical texture-space ray traversal with custom Vulkan intersection shaders}
      \resumeItem{Integrated techniques into complete Monte Carlo path tracing pipeline supporting glTF 2.0 scenes, HDR environment maps, and KHR\_materials\_displacement extension for physically-based rendering}
      \resumeItem{GitHub Repo: \href{https://github.com/MatForge/MatForge}{\uline{github.com/MatForge/MatForge}}}
      \resumeItemListEnd

\resumeProjectHeading
      {\textbf{Vulkan Grass Renderer} $|$ \emph{Vulkan, GLSL, C++}}{Oct 2025 -- Nov 2025}
      \resumeItemListStart
      \resumeItem{Implemented real-time grass rendering system in Vulkan, simulating up to 1 million grass blades with physics-based animation at interactive frame rates using compute and tessellation pipelines}
      \resumeItem{Developed GPU compute shader for physics simulation applying gravity, Hooke's law recovery forces, and wind with spatial turbulence to quadratic Bezier curve grass blades, processing 1M+ blade state updates per frame}
      \resumeItem{Engineered three-tiered GPU culling system (orientation, view-frustum, distance-based probabilistic) achieving 4.31$\times$ performance improvement at 1M blades through efficient atomic operations and indirect draw calls}
      \resumeItem{Built hardware tessellation pipeline with dynamic LOD based on camera distance, generating smooth blade geometry with exponential falloff from 10 to 2 subdivisions for optimal geometry density}
      \resumeItem{Created comprehensive ImGUI control panel with real-time FPS metrics (average, 1\% lows), frame time graphs, and automated performance testing system generating CSV benchmarks across 40 configurations}
      \resumeItem{GitHub Repo: \href{https://github.com/tonytgrt/Vulkan-Grass-Renderer}{\uline{github.com/tonytgrt/Vulkan-Grass-Renderer}}}
      \resumeItemListEnd

\resumeProjectHeading
  {\textbf{CUDA Path Tracer} $|$ \emph{CUDA, GLSL, C++}}{Sep 2025 - Oct 2025}
  \resumeItemListStart
  \resumeItem{Monte Carlo path tracer capable of rendering complex 3D scenes with custom 3D models and environment maps}
  \resumeItem{Implemented shading BSDF kernel supporting global illumination, multiple importance sampling, anti-aliasing, sub-surface scattering, capable of rendering various PBR material types with albedo and texture maps}
  \resumeItem{Integrated third-party libraries of tinyGLTF to support glTF 2.0 mesh loading and Nvidia OptiX for denoising}
  \resumeItem{Utilized various techniques to boost performance: material sorting (+5\%), Russian Roulette (+6\% - 24\%), stream compaction (+24\% - 67\%), and Bounding Volume Hierachy (3$\times$ - $160\times$ framerate in complex scenes) }
  \resumeItem{Project Repo and Demo: \href{https://github.com/tonytgrt/CUDA-Path-Tracer}{\uline{github.com/tonytgrt/CUDA-Path-Tracer}}. A previous standalone performance focused stream compaction project with detailed analysis in Nsight: \href{https://github.com/tonytgrt/Project2-Stream-Compaction}{\uline{github.com/tonytgrt/Project2-Stream-Compaction}}}
\resumeItemListEnd

\resumeProjectHeading
      {\textbf{WebGPU Renderer} $|$ \emph{WebGPU, TypeScript, WGSL}}{Sep 2025 -- Oct 2025}
      \resumeItemListStart
      \resumeItem{Implemented three advanced rendering techniques for real-time lighting of 5000+ dynamic point lights: Naive Forward, Forward+, and Clustered Deferred rendering using WebGPU compute and graphics pipelines}
      \resumeItem{Engineered screen-space light clustering system using compute shaders, subdividing view frustum into 16×9×24 grid with exponential depth slicing and sphere-AABB intersection testing for efficient light culling}
      \resumeItem{Developed G-buffer architecture with 3 render targets (position, normal, albedo) enabling two-pass deferred rendering that decouples geometry complexity from lighting calculations}
      \resumeItem{Achieved 53x performance improvement over naive rendering (497ms to 9.3ms at 5000 lights) and 3.5x speedup vs Forward+ through overdraw elimination and optimized memory access patterns}
      \resumeItem{Built automated performance testing system collecting statistical frame time data across 30 configurations (3 renderers × 10 light counts), generating CSV analysis for rigorous benchmarking}
      \resumeItem{Live demo deployed at \href{http://webgpu.tonyxtian.com}{\uline{webgpu.tonyxtian.com}}, rendering Sponza scene with real-time performance metrics and interactive controls. GitHub repo: \href{https://github.com/tonytgrt/Project4-WebGPU-Forward-Plus-and-Clustered-Deferred}{\uline{github.com/tonytgrt/Project4-WebGPU-Forward-Plus-and-Clustered-Deferred}}}
      \resumeItemListEnd
    

      
\resumeSubHeadingListEnd




\enlargethispage{20pt}

%-----------PROGRAMMING SKILLS-----------
\section{Technical Skills}
 \begin{itemize}[leftmargin=0.15in, label={}]
    \small{\item{
    \textbf{Programming}{: C++, CUDA, Python, WGSL, GLSL, Parallel algorithms, Memory management, Rendering pipeline} \\
    \textbf{Graphics/Rendering}{: NSight Profiling, Path Tracing, Deferred Rendering, Rasterization, Animation systems, PBR} \\
    \textbf{Tools/APIs}{: Nvidia NSight, WebGPU, Vulkan, Visual Studio, Qt, OpenGL, Git, CMake, MakeFile, Clang, GDB}
    }}
 \end{itemize}

%-------------------------------------------
\end{document}