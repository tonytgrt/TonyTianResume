%-------------------------
% Resume in Latex
% Author : Jake Gutierrez
% Based off of: https://github.com/sb2nov/resume
% License : MIT
%------------------------

\documentclass[letterpaper,11pt]{article}

\usepackage{latexsym}
\usepackage[empty]{fullpage}
\usepackage{titlesec}
\usepackage{marvosym}
\usepackage[usenames,dvipsnames]{color}
\usepackage{verbatim}
\usepackage{enumitem}
\usepackage[hidelinks]{hyperref}
\usepackage{fancyhdr}
\usepackage[english]{babel}
\usepackage{tabularx}
\input{glyphtounicode}



%----------FONT OPTIONS----------
% sans-serif
% \usepackage[sfdefault]{FiraSans}
% \usepackage[sfdefault]{roboto}
% \usepackage[sfdefault]{noto-sans}
% \usepackage[default]{sourcesanspro}

% serif
% \usepackage{CormorantGaramond}
% \usepackage{charter}


\pagestyle{fancy}
\fancyhf{} % clear all header and footer fields
\fancyfoot{}
\renewcommand{\headrulewidth}{0pt}
\renewcommand{\footrulewidth}{0pt}

% Adjust margins
\addtolength{\oddsidemargin}{-0.5in}
\addtolength{\evensidemargin}{-0.7in}
\addtolength{\textwidth}{1in}
\addtolength{\topmargin}{-0.75in}
\addtolength{\textheight}{1.0in}


\urlstyle{same}

%\raggedbottom
\raggedright
\setlength{\tabcolsep}{0in}

% Sections formatting
\titleformat{\section}{
  \vspace{-10pt}\scshape\raggedright\large
}{}{0em}{}[\color{black}\titlerule \vspace{-5pt}]

% Ensure that generate pdf is machine readable/ATS parsable
\pdfgentounicode=1

%-------------------------
% Custom commands
\newcommand{\resumeItem}[1]{
  \item\small{
    {#1 \vspace{-3pt}}
  }
}

\newcommand{\resumeSubheading}[4]{
  \vspace{-2pt}\item
    \begin{tabular*}{0.97\textwidth}[t]{l@{\extracolsep{\fill}}r}
      \textbf{#1} & #2 \\
      \textit{\small#3} & \textit{\small #4} \\
    \end{tabular*}\vspace{-7pt}
}

\newcommand{\resumeSubSubheading}[2]{
    \vspace{-7pt}\item
    \begin{tabular*}{0.97\textwidth}{l@{\extracolsep{\fill}}r}
      \textit{\small#1} & \textit{\small #2} \\
    \end{tabular*}\vspace{-7pt}
}

\newcommand{\resumeProjectHeading}[2]{
    \item
    \begin{tabular*}{0.97\textwidth}{l@{\extracolsep{\fill}}r}
      \small#1 & #2 \\
    \end{tabular*}\vspace{-7pt}
}

\newcommand{\resumeSubItem}[1]{\resumeItem{#1}\vspace{-4pt}}

\renewcommand\labelitemii{$\vcenter{\hbox{\tiny$\bullet$}}$}

\newcommand{\resumeSubHeadingListStart}{\begin{itemize}[leftmargin=0.15in, label={}]}
\newcommand{\resumeSubHeadingListEnd}{\end{itemize}}
\newcommand{\resumeItemListStart}{\begin{itemize}}
\newcommand{\resumeItemListEnd}{\end{itemize}\vspace{-5pt}}

%-------------------------------------------
%%%%%%  RESUME STARTS HERE  %%%%%%%%%%%%%%%%%%%%%%%%%%%%


\begin{document}

%----------HEADING----------
\begin{center}
    \textbf{\Huge \scshape Tony Yiding Tian} \\ \vspace{1pt}
    \small Philadelphia, PA $|$ (267)249-1202 $|$ \href{mailto:tonytg@seas.upenn.edu}{\underline{tonytg@seas.upenn.edu}} $|$ 
    \href{https://tonyxtian.com}{\underline{tonyxtian.com}}
\end{center}

%-----------EDUCATION-----------
\section{Education}
  \resumeSubHeadingListStart
    \resumeSubheading
      {University of Pennsylvania - School of Engineering and Applied Sciences}{Philadelphia, PA}
      {B.S.E. in Computer Engineering, Accelerated M.S.E. in Computer Graphics and Game Technology}{May 2027}
      \resumeItemListStart
        \resumeItem{GPA: 3.74 | Relevant Courses: GPU Programming, Advanced Rendering, Interactive Computer Graphics, Computer Animation, Operating Systems, Data Structures \& Algorithms, Computer Architecture}
      \resumeItemListEnd
  \resumeSubHeadingListEnd

  %-----------PROJECTS-----------
\section{Projects}
\resumeSubHeadingListStart

\resumeProjectHeading
  {\textbf{CUDA Path Tracer} $|$ \emph{CUDA, C++}}{Sep 2025 - Oct 2025}
  \resumeItemListStart
  \resumeItem{Monte Carlo path tracer capable of rendering complex 3D scenes with custom 3D models and environment maps}
  \resumeItem{Implemented shading BSDF kernel supporting global illumination, multiple importance sampling, anti-aliasing, sub-surface scattering, capable of rendering various PBR material types with albedo and texture maps}
  \resumeItem{Integrated third-party libraries of tinyGLTF to support glTF 2.0 mesh loading and Nvidia OptiX for denoising}
  \resumeItem{Utilized various techniques to boost performance: material sorting (+5\%), Russian Roulette (+6\% - 24\%), stream compaction (+24\% - 67\%), and Bounding Volume Hierachy (3$\times$ - $160\times$ framerate in complex scenes). }
  \resumeItem{Project Repo and Demo: \href{https://github.com/tonytgrt/CUDA-Path-Tracer}{github.com/tonytgrt/CUDA-Path-Tracer}. A previous standalone performance focused stream compaction project with detailed analysis in Nsight: \href{https://github.com/tonytgrt/Project2-Stream-Compaction}{github.com/tonytgrt/Project2-Stream-Compaction}.}
\resumeItemListEnd
      
\resumeProjectHeading
      {\textbf{Mini Minecraft - Voxel-based 3D Game} $|$ \emph{C++, GLSL, Qt}}{Oct 2024 -- Dec 2024}
      \resumeItemListStart
      \resumeItem{Collaborated in team of 3 to develop fully-featured voxel game engine in C++ using OpenGL, generating infinite worlds with 1M+ blocks and maintaining 60+ FPS performance}
      \resumeItem{Engineered procedural terrain generation system using layered 2D/3D Perlin noise algorithms, creating 5 distinct biomes (Grassland, Mountain, Desert, Islands, Caves) with biome-specific block distributions and procedurally placed vegetation assets}
      \resumeItem{Implemented post-processing rendering pipeline with custom GLSL fragment shaders, featuring dynamic underwater/lava distortion effects using UV coordinate manipulation and real-time crosshair overlay rendering}
      \resumeItem{Developed dual physics simulation system: gravity-based collision detection with terrain for ground movement, and buoyancy calculations for water/lava interaction, plus creative fly-mode with 6-DOF movement}
      \resumeItem{Built efficient chunk-based world management system with frustum culling and LOD optimization, reducing draw calls by 80\% through face culling of adjacent blocks}
      \resumeItem{Implemented real-time block manipulation (mining/placing) with ray-casting intersection testing and immediate mesh updates, supporting 16 different block types with unique textures and properties}
      \resumeItem{Project demo showcasing all features: \href{https://youtu.be/jRb4EHV5KQI}{youtu.be/jRb4EHV5KQI}}
      \resumeItemListEnd

      \resumeProjectHeading
      {\textbf{PennOS - UNIX-like Operating System} $|$ \emph{C, Shell, Kernel}}{Mar 2025 -- May 2025}
      \resumeItemListStart
        \resumeItem{Architected and implemented a complete user-level operating system in C with team of 4, featuring 8000+ lines of systems code with full process lifecycle management}
        \resumeItem{Designed Process Control Block (PCB) data structure managing 50+ concurrent processes with metadata including PID allocation, priority levels, parent-child relationships, signal handling, and user/kernel stack management}
        \resumeItem{Implemented preemptive multi-level priority scheduler supporting 3 priority levels with Round Robin time-slicing (10ms quantum), preventing starvation through priority aging and achieving 95\% CPU utilization}
        \resumeItem{Built POSIX-compliant interactive shell supporting 15+ built-in commands (ps, kill, jobs, fg/bg), I/O redirection, pipeline chaining, and batch script execution with robust error handling}
      \resumeItemListEnd
      
\resumeSubHeadingListEnd

%-----------EXPERIENCE-----------
\section{Experience}
  \resumeSubHeadingListStart
    \resumeSubheading
      {Linux Kernel Policies Research Assistant - PURM Scholar}{May 2025 -- Aug 2025}
      {Learning Directed Operating System (\href{https://ldos.utexas.edu/}{\underline{LDOS}}), Prof. Sebastian Angel}{Philadelphia, PA}
      \resumeItemListStart
        \resumeItem{Developed eBPF-based kernel monitoring infrastructure collecting real-time TCP networking metrics retrieved from 5 crucial kernel tcp functions of tcp\_v4\_rcv, v4\_connect, state\_process, congestion\_control, and cubic}
        \resumeItem{Engineered high-performance data analysis pipeline processing 10,000+ TCP state transitions per second, identifying critical performance bottlenecks in kernel networking policies and congestion control algorithms}
        \resumeItem{Architected and contributed 2000+ lines of C and Python code to open-source KernMLOps repo, implementing kernel probing infrastructure used by 15+ researchers. Repo link: \href{https://github.com/tonytgrt/KernMLOps}{github.com/tonytgrt/KernMLOps}}
      \resumeItemListEnd

  \resumeSubHeadingListEnd



\enlargethispage{20pt}

%-----------PROGRAMMING SKILLS-----------
\section{Technical Skills}
 \begin{itemize}[leftmargin=0.15in, label={}]
    \small{\item{
    \textbf{Graphics/Rendering}{: NSight Profiling, Path Tracing, Deferred Rendering, Rasterization, Animation systems, PBR} \\
    \textbf{Programming}{: CUDA, C/C++, GLSL, Parallel algorithms, Memory management, Rendering pipeline} \\
    \textbf{Tools/APIs}{: Nvidia NSight, Visual Studio, Qt, OpenGL, WebGPU, Git, CMake, MakeFile, Clang, GDB, GCC}
    }}
 \end{itemize}

%-------------------------------------------
\end{document}