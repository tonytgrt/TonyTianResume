%%%%%%%%%%%%%%%%%%%%%%%%%%%%%%%%%%%%%%%%%%%%%%%%%%%
\name{田一丁}
\contactInfo{+1 267-249-1202}{tonytg@seas.upenn.edu}{https://tonyxtian.com}
%%%%%%%%%%%%%%%%%%%%%%%%%%%%%%%%%%%%%%%%%%%%%%%%%%%

%%%%%%%%%%%%%%%%%%%%%%%%%%%%%%%%%%%%%%%%%%%%%%%%%%%
\logosection{\faGraduationCap}{教育经历}

\datedline{\textbf{University of Pennsylvania \quad 宾夕法尼亚大学}}{\dateRange{2023.08}{2027.05}}

Computer Engineering, BSE  \: 计算机工程,本科

Computer Graphics, MSE  \: 计算机图形,加速硕士
\begin{itemize}
  \item GPA: 3.78
  \item 相关课程:GPU编程,3D建模,计算机动画,UI/UX设计,机器学习,操作系统实现,数据结构与算法
\end{itemize}

%%%%%%%%%%%%%%%%%%%%%%%%%%%%%%%%%%%%%%%%%%%%%%%%%%%

%%%%%%%%%%%%%%%%%%%%%%%%%%%%%%%%%%%%%%%%%%%%%%%%%%%
\logosection{\faWrench}{项目经历}

\datedline{\textbf{MatForge - 高级材质渲染系统} $|$ \emph{Vulkan, Slang, C++}}{\dateRange{2025.11}{2025.12}}
\begin{itemize}
  \item 设计生产级GPU路径追踪器,在统一Vulkan光线追踪管线中实现四篇SIGGRAPH论文(2023-2024),贡献2400+行C++和Slang代码用于采样和几何系统
  \item 实现SIGGRAPH 2024的四元优化低差异序列(QOLDS),设计47维Owen扰动的Base-3 Sobol生成器,在512 SPP下相比传统PCG采样提升+2.57 dB PSNR,MSE降低44.7\%,性能开销$<$1\%
  \item 开发RMIP(矩形最小最大图像金字塔)相交着色器,通过分层纹理空间光线遍历和自定义Vulkan相交着色器实现无需细分的位移贴图
  \item 将技术集成到完整的蒙特卡洛路径追踪管线,支持glTF 2.0场景、HDR环境贴图和KHR材料扩展
  \item 项目仓库: \url{https://github.com/MatForge/MatForge}
\end{itemize}

\datedline{\textbf{Vulkan草地渲染器 - 基于物理的草地模拟} $|$ \emph{Vulkan, GLSL, C++}}{\dateRange{2025.10}{2025.11}}
\begin{itemize}
  \item 在Vulkan中实现实时草地渲染系统,使用计算和曲面细分管线以交互帧率模拟多达100万根草叶的物理动画
  \item 开发GPU计算着色器进行物理模拟,对二次贝塞尔曲线草叶应用重力、胡克定律恢复力和带空间湍流的风力,每帧处理100万+草叶状态更新
  \item 设计三级GPU剔除系统(朝向、视锥、基于距离的概率剔除),在100万草叶时通过高效原子操作和间接绘制调用实现4.31倍性能提升
  \item 构建硬件曲面细分管线,基于相机距离动态LOD,从10到2细分的指数衰减生成平滑草叶几何体
  \item 项目仓库: \url{https://github.com/tonytgrt/Vulkan-Grass-Renderer}
\end{itemize}

\datedline{\textbf{CUDA路径追踪器 - 3D PBR渲染器} $|$ \emph{CUDA, GLSL, C++}}{\dateRange{2025.09}{2025.10}}
\begin{itemize}
  \item 蒙特卡洛路径追踪器,能够渲染具有自定义3D模型和环境贴图的复杂3D场景
  \item 实现支持全局光照、多重重要性采样、抗锯齿、次表面散射的着色BSDF内核,能够渲染各种带反照率和纹理贴图的PBR材质类型
  \item 集成tinyGLTF第三方库支持glTF 2.0网格加载,以及Nvidia OptiX用于降噪
  \item 使用多种技术提升性能:材质排序(+5\%)、俄罗斯轮盘(+6\% - 24\%)、流压缩(+24\% - 67\%)、层次包围体(复杂场景3倍 - 160倍帧率)
  \item 项目仓库: \url{https://github.com/tonytgrt/CUDA-Path-Tracer}
\end{itemize}

\datedline{\textbf{WebGPU渲染器 - Web实时渲染器} $|$ \emph{WebGPU, TypeScript, WGSL}}{\dateRange{2025.09}{2025.10}}
\begin{itemize}
  \item 使用WebGPU计算和图形管线实现三种高级渲染技术用于5000+动态点光源的实时光照:朴素前向、Forward+和聚簇延迟渲染
  \item 使用计算着色器设计屏幕空间光源聚簇系统,将视锥体细分为16×9×24网格,使用指数深度切片和球体-AABB相交测试进行高效光源剔除
  \item 开发具有3个渲染目标(位置、法线、反照率)的G-buffer架构,实现将几何复杂度与光照计算解耦的两遍延迟渲染
  \item 相比传统渲染实现53倍性能提升(5000光源下从497ms降至9.3ms),通过消除过度绘制和优化内存访问模式比Forward+快3.5倍
  \item 在线演示: \url{https://webgpu.tonyxtian.com},项目仓库: \url{https://github.com/tonytgrt/Project4-WebGPU-Forward-Plus-and-Clustered-Deferred}
\end{itemize}

%%%%%%%%%%%%%%%%%%%%%%%%%%%%%%%%%%%%%%%%%%%%%%%%%%%

%%%%%%%%%%%%%%%%%%%%%%%%%%%%%%%%%%%%%%%%%%%%%%%%%%%
\logosection{\faCogs}{相关技能}
\begin{itemize}[parsep=0.5ex]
  \item 编程语言: C++, CUDA, Python, WGSL, GLSL, 并行算法, 内存管理, 渲染管线
  \item 图形/渲染: NSight性能分析, 路径追踪, 延迟渲染, 光栅化, 动画系统, PBR
  \item 工具/API: Unreal, Unity, Maya, Nvidia NSight, WebGPU, Vulkan, Visual Studio, Qt, OpenGL, Git, CMake
  \item 语言:英语 - 熟练(TOEFL 110),粤语 - 入门,日语 - 入门
\end{itemize}
%%%%%%%%%%%%%%%%%%%%%%%%%%%%%%%%%%%%%%%%%%%%%%%%%%%

