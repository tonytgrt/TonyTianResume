%%%%%%%%%%%%%%%%%%%%%%%%%%%%%%%%%%%%%%%%%%%%%%%%%%%
\name{田一丁}
\contactInfo{+1 267-249-1202}{tonytg@seas.upenn.edu}{https://tonyxtian.com}
%%%%%%%%%%%%%%%%%%%%%%%%%%%%%%%%%%%%%%%%%%%%%%%%%%%

%%%%%%%%%%%%%%%%%%%%%%%%%%%%%%%%%%%%%%%%%%%%%%%%%%%
\logosection{\faGraduationCap}{教育经历}

\datedline{\textbf{宾夕法尼亚大学 \quad University of Pennsylvania}}{2027届本硕连读}

\datedline{计算机工程\:本科 (Computer Engineering, BSE)}{\dateRange{2023.08}{2027.05}}
\datedline{计算机图形\:加速硕士 (Computer Graphics, MSE)}{\dateRange{2025.01}{2027.05}}

\begin{itemize}
  \item GPA 3.78
  \item 相关课程: GPU编程、游戏设计、3D建模、计算机动画、UI/UX设计、机器学习、操作系统、数据结构与算法
  \item 到岗时间:2026/05/25 - 2026/08/25
\end{itemize}

%%%%%%%%%%%%%%%%%%%%%%%%%%%%%%%%%%%%%%%%%%%%%%%%%%%

%%%%%%%%%%%%%%%%%%%%%%%%%%%%%%%%%%%%%%%%%%%%%%%%%%%
\logosection{\faLaptopCode}{项目经历}
\datedline{\textbf{CUDA路径追踪器 - 3D PBR渲染器} $|$ \emph{CUDA, GLSL, C++}}{\dateRange{2025.09}{2025.10}}
\begin{itemize}
  \item 蒙特卡洛路径追踪器,可渲染包含自定义模型与环境贴图的复杂3D场景
  \item 实现BSDF Shading Kernel,支持全局光照、多重重要性采样(MIS)、抗锯齿、次表面散射,可渲染多种带Albedo与纹理贴图的PBR材质
  \item 集成了tinyGLTF库以支持glTF 2.0的模型加载,并使用Nvidia OptiX进行AI降噪
  \item 实现多项性能优化:材质排序(+5\%)、Russian Roulette(+6\%\textasciitilde24\%)、Stream Compaction(+24\%\textasciitilde67\%)、BVH加速结构(复杂场景帧率提升3\textasciitilde160倍)
  \item 项目仓库: \url{https://github.com/tonytgrt/CUDA-Path-Tracer}
\end{itemize}

\datedline{\textbf{WebGPU渲染器 - 网页端GPU实时渲染器} $|$ \emph{WebGPU, TypeScript, WGSL}}{\dateRange{2025.10}{2025.11}}
\begin{itemize}
  \item 基于WebGPU Compute与图形管线,在5000余个动态光源的场景中实现三种实时光照方案:Naive Forward、Forward+、Clustered Deferred
  \item 设计屏幕空间光源聚簇系统:将视锥体划分为16×9×24的网格,采用指数深度切片与球体-AABB相交检测以高效剔除光源
  \item 构建三目标G-buffer架构(位置、法线、Albedo),实现两遍延迟渲染,将几何复杂度与光照计算解耦
  \item Clustered Deferred消除了过度绘制与优化了内存访问,性能较Naive Forward最高提升53倍,较Forward+提升3.5倍
  \item 在线演示: \url{https://webgpu.tonyxtian.com},可使用任何支持WebGPU的设备无需下载实时体验Demo
  \item 项目仓库: \url{https://github.com/tonytgrt/Project4-WebGPU-Forward-Plus-and-Clustered-Deferred}
\end{itemize}

\datedline{\textbf{Mini Minecraft} $|$ \emph{C++, Qt, GLSL}}{\dateRange{2024.10}{2024.12}}
\begin{itemize}
  \item 使用C++和OpenGL开发完整的游戏引擎,生成含100万+方块的无限世界并保持60+ FPS
  \item 构建程序化地形生成系统,使用分层2D/3D柏林噪声算法,创建5种不同生物群落(草地、山脉、沙漠、岛屿、洞穴),含生物群落特定方块分布和程序化植被生成
  \item 实现后处理渲染管线与自定义GLSL片段着色器,支持水下/岩浆动态扭曲效果及实时准星叠加渲染
  \item 开发双物理模拟系统:基于重力的地面碰撞检测,以及水/岩浆浮力计算,另含6自由度创造模式飞行
  \item 实现实时方块操作(挖掘/放置),基于光线投射求交检测与即时网格更新,支持16种不同方块类型
  \item 项目演示: https://youtu.be/jRb4EHV5KQI
\end{itemize}

\logosection{\faFlask}{科研经历}

\datedline{\textbf{学习导向操作系统(LDOS)}}{\dateRange{2025.05}{2025.08}}
\begin{itemize}
  \item 开发基于eBPF的内核监控基础设施,从tcp\_v4\_rcv、v4\_connect、state\_process、congestion\_control等关键内核TCP函数中采集实时网络指标
  \item 构建高性能数据分析管线,每秒处理10,000+次TCP状态转换,识别内核网络策略与拥塞控制算法中的关键性能瓶颈
  \item 向开源KernMLOps仓库贡献2000+行C和Python代码,实现的内核探测基础设施被15余名研究人员使用。项目仓库: \url{https://github.com/tonytgrt/KernMLOps}
\end{itemize}

%%%%%%%%%%%%%%%%%%%%%%%%%%%%%%%%%%%%%%%%%%%%%%%%%%%

%%%%%%%%%%%%%%%%%%%%%%%%%%%%%%%%%%%%%%%%%%%%%%%%%%%
\logosection{\faCode}{相关技能}
\begin{itemize}[parsep=0.5ex]
  \item 编程: C++, CUDA, Python, WGSL, GLSL;并行算法, 内存管理, 渲染管线开发
  \item 图形/渲染: NSight性能分析, 路径追踪, 延迟渲染, 光栅化, 动画系统, PBR
  \item 工具/API: Unreal, Unity, Maya, Nvidia NSight, WebGPU, Vulkan, Visual Studio, Qt, OpenGL, Git, CMake
  %\item 语言: 英语流利(托福110),日语入门
\end{itemize}
%%%%%%%%%%%%%%%%%%%%%%%%%%%%%%%%%%%%%%%%%%%%%%%%%%%

