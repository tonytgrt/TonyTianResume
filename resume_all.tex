%-------------------------
% Resume in Latex
% Author : Jake Gutierrez
% Based off of: https://github.com/sb2nov/resume
% License : MIT
%------------------------

\documentclass[letterpaper,11pt]{article}

\usepackage{latexsym}
\usepackage[empty]{fullpage}
\usepackage{titlesec}
\usepackage{marvosym}
\usepackage[usenames,dvipsnames]{color}
\usepackage{verbatim}
\usepackage{enumitem}
\usepackage[hidelinks]{hyperref}
\usepackage{fancyhdr}
\usepackage[english]{babel}
\usepackage{tabularx}
\input{glyphtounicode}



%----------FONT OPTIONS----------
% sans-serif
% \usepackage[sfdefault]{FiraSans}
% \usepackage[sfdefault]{roboto}
% \usepackage[sfdefault]{noto-sans}
% \usepackage[default]{sourcesanspro}

% serif
% \usepackage{CormorantGaramond}
% \usepackage{charter}


\pagestyle{fancy}
\fancyhf{} % clear all header and footer fields
\fancyfoot{}
\renewcommand{\headrulewidth}{0pt}
\renewcommand{\footrulewidth}{0pt}

% Adjust margins
\addtolength{\oddsidemargin}{-0.5in}
\addtolength{\evensidemargin}{-0.7in}
\addtolength{\textwidth}{1in}
\addtolength{\topmargin}{-0.75in}
\addtolength{\textheight}{1.0in}


\urlstyle{same}

%\raggedbottom
\raggedright
\setlength{\tabcolsep}{0in}

% Sections formatting
\titleformat{\section}{
  \vspace{-10pt}\scshape\raggedright\large
}{}{0em}{}[\color{black}\titlerule \vspace{-5pt}]

% Ensure that generate pdf is machine readable/ATS parsable
\pdfgentounicode=1

%-------------------------
% Custom commands
\newcommand{\resumeItem}[1]{
  \item\small{
    {#1 \vspace{-3pt}}
  }
}

\newcommand{\resumeSubheading}[4]{
  \vspace{-2pt}\item
    \begin{tabular*}{0.97\textwidth}[t]{l@{\extracolsep{\fill}}r}
      \textbf{#1} & #2 \\
      \textit{\small#3} & \textit{\small #4} \\
    \end{tabular*}\vspace{-7pt}
}

\newcommand{\resumeSubSubheading}[2]{
    \vspace{-7pt}\item
    \begin{tabular*}{0.97\textwidth}{l@{\extracolsep{\fill}}r}
      \textit{\small#1} & \textit{\small #2} \\
    \end{tabular*}\vspace{-7pt}
}

\newcommand{\resumeProjectHeading}[2]{
    \item
    \begin{tabular*}{0.97\textwidth}{l@{\extracolsep{\fill}}r}
      \small#1 & #2 \\
    \end{tabular*}\vspace{-7pt}
}

\newcommand{\resumeSubItem}[1]{\resumeItem{#1}\vspace{-4pt}}

\renewcommand\labelitemii{$\vcenter{\hbox{\tiny$\bullet$}}$}

\newcommand{\resumeSubHeadingListStart}{\begin{itemize}[leftmargin=0.15in, label={}]}
\newcommand{\resumeSubHeadingListEnd}{\end{itemize}}
\newcommand{\resumeItemListStart}{\begin{itemize}}
\newcommand{\resumeItemListEnd}{\end{itemize}\vspace{-5pt}}

%-------------------------------------------
%%%%%%  RESUME STARTS HERE  %%%%%%%%%%%%%%%%%%%%%%%%%%%%


\begin{document}

%----------HEADING----------
% \begin{tabular*}{\textwidth}{l@{\extracolsep{\fill}}r}
%   \textbf{\href{http://sourabhbajaj.com/}{\Large Sourabh Bajaj}} & Email : \href{mailto:sourabh@sourabhbajaj.com}{sourabh@sourabhbajaj.com}\\
%   \href{http://sourabhbajaj.com/}{http://www.sourabhbajaj.com} & Mobile : +1-123-456-7890 \\
% \end{tabular*}

\begin{center}
    \textbf{\Huge \scshape Tony Yiding Tian} \\ \vspace{1pt}
    \small Philadelphia, PA $|$ (267)249-1202 $|$ \href{mailto:tonytg@seas.upenn.edu}{\underline{tonytg@seas.upenn.edu}} $|$ 
    \href{https://github.com/tonytgrt}{\underline{github.com/tonytgrt}}
    %\href{https://github.com/...}{\underline{github.com/jake}}
\end{center}


%-----------EDUCATION-----------
\section{Education}
  \resumeSubHeadingListStart
    \resumeSubheading
      {University of Pennsylvania - School of Engineering and Applied Sciences}{Philadelphia, PA}
      {Bachelor of Engineering in Computer Engineering}{May 2027}
    \resumeSubSubheading
      {Master of Engineering in Computer Graphics and Game Technology}{May 2027}
      \resumeItemListStart
        \resumeItem{Grade/Credit: GPA 3.74 / 26 Credit Hours}
        \resumeItem{Relevant Courses: GPU Programming, Advanced Rendering, Interactive Computer Graphics, Operating Systems Design \& Implementation, Data Structure \& Algorithm, Embedded Systems, Electrical Circuits \& Systems}
      \resumeItemListEnd
  \resumeSubHeadingListEnd


  %-----------EXPERIENCE-----------
\section{Experience}
  \resumeSubHeadingListStart
    \resumeSubheading
      {Learning Directed Operating System (\href{https://ldos.utexas.edu/}{\underline{LDOS}}), Researcher}{May 2025 -- Present}
      {Distributed Systems Laboratory at UPenn}{Philadelphia, PA}
      \resumeItemListStart
        \resumeItem{Penn Undergraduate Research Mentorship (\href{https://curf.upenn.edu/content/penn-undergraduate-research-mentoring-program-purm}{\underline{PURM}}) 2025 Scholarship awardee, student of \href{https://www.cis.upenn.edu/~sga001/}{\underline{Prof. Sebastian Angel}}.}
        \resumeItem{Contributed to the KernMLOps repository, a tool that probes into the Linux Kernel and collect data about the kernel's decision making policies in real-time: \href{https://github.com/tonytgrt/KernMLOps/tree/main}{github.com/tonytgrt/KernMLOps/tree/main}.}
        \resumeItem{Used eBPF kprobes to collect data of the TCP networking traffic subsystem across various workloads, including the iperf3 and the redis benchmark.}
        \resumeItem{Obtained data of a distribution of different TCP states of each traffic entry, analyzed data in Python.}
    \resumeItemListEnd

    \resumeSubheading
      {Social Media Mobile App Feature Researcher \& Designer}{Jan 2025 -- May 2025}
      {YesTech, Corp.}{Remote}
      \resumeItemListStart
        \resumeItem{Worked in user-centric team of the startup company to develop a social media app Best Friends Network.}
        \resumeItem{Researched and designed Friendship Portal, a core feature of the app that encourages friends to share their moods.}
        \resumeItem{Tested various app features, designed and distributed surveys, and provided constructive feedback to developers.}
    \resumeItemListEnd

    \resumeSubheading
      {Guest Service Attendant}{May 2024 -- Aug 2025}
      {Residential and Hospitality Services at UPenn}{Philadelphia, PA}
      \resumeItemListStart
        \resumeItem{Facilitated the move-in of 1000+ high school students over the summers.}
        \resumeItem{Managed important access credentials of the building for residents and facility workers.}
        \resumeItem{Distributed essential college house resources to residents in need (temporary credentials, carts, linens).}
        \resumeItem{Communicated frequently with residents including minors, delivering excellent customer service to the guests.}
      \resumeItemListEnd

    \resumeSubheading
      {Genshin Impact, Quality Inspector}{Dec 2022 -- Oct 2023}
      {miHoYo}{Remote}
      \resumeItemListStart
        \resumeItem{Participated in the open-world mobile role-play game Genshin Impact's beta tests before the release of each major update every 6 weeks.}
        \resumeItem{Quality assurance of the game's product: characters eligible of in-game purchases. Provided thorough assessment of 20+ characters' performance and design before their release, each character harnessing over \$1M of sales profit.}
        \resumeItem{Tested the open-world experience of combats and puzzles in the game. Discovered and reported bugs in gameplay.}
    \resumeItemListEnd
      
% -----------Multiple Positions Heading-----------
%    \resumeSubSubheading
%     {Software Engineer I}{Oct 2014 - Sep 2016}
%     \resumeItemListStart
%        \resumeItem{Apache Beam}
%          {Apache Beam is a unified model for defining both batch and streaming data-parallel processing pipelines}
%     \resumeItemListEnd
%    \resumeSubHeadingListEnd
%-------------------------------------------

    


  \resumeSubHeadingListEnd


  
%-----------PROJECTS-----------
\section{Projects}
    \resumeSubHeadingListStart
    
    \resumeProjectHeading
          {\textbf{T\&T Slots - Sense the Win} $|$ \emph{ATMega328PB, SPO2 Sensor, I2C, Embedded C}}{Mar 2025 -- May 2025}
          \resumeItemListStart
            \resumeItem{Built an electronic slot machine that takes into account the player's heart rate to adjust the probability of winning, developing core software features:}
            \resumeItem{Graphical gameplay GUI on a LCD display connected to the Microchip ATMega328PB, developed the graphical library for the embedded system.}
            \resumeItem{Communication of the MAX30102 SPO2 heart rate sensor using I2C, developed an sensor-specific I2C library.}
            \resumeItem{Sound effects via a buzzer using PWM digital output.}
            \resumeItem{Links: \href{https://upenn-embedded.github.io/final-project-s25-t-t-slots-sense-the-win/}{\underline{Project Website}}} \href{https://github.com/upenn-embedded/final-project-s25-t-t-slots-sense-the-win}{\underline{Github repo}}
          \resumeItemListEnd
          
    \resumeProjectHeading
          {\textbf{PennOS} $|$ \emph{Priority Scheduler, Interactive Shell, C}}{Mar 2025 -- May 2025}
          \resumeItemListStart
            \resumeItem{Implemented a user-level UNIX-like operating system using C in group of four, developing core kernel features:}
            \resumeItem{Process Control Block (PCB) data structure that stores metadata of a process including PID, priority, parent/child relationships, process state, and stack/thread info.}
            \resumeItem{Multi-level priority scheduler that distributes CPU resources to processes according to their priority level. Processes of same priority are scheduled with Round Robin without any starvation.}
            \resumeItem{Interactive shell that supports common UNIX commands, I/O redirection, and script execution.}
            \resumeItem{System calls for spawning processes, waiting, sleeping, signal delivery, process exit, and file I/O wrappers.}
          \resumeItemListEnd
          
    \resumeProjectHeading
          {\textbf{Mini Minecraft} $|$ \emph{OpenGL, Qt, C++, GLSL}}{Oct 2024 -- Dec 2024}
          \resumeItemListStart
            \resumeItem{Participated in a group of three to implement Minecraft using C++ and GLSL in Qt, developing core features:}
            \resumeItem{Procedural terrain generation using 2D and 3D perlin noise, including Grassland, Mountain, Desert, Islands, and Cave. Each terrain has unique geographical traits and procedurally placed assets such as trees and cacti.}
            \resumeItem{Post-process shaders with GLSL. Set up a post-process rendering pipeline with the OpenGL graphical API that applies a distortion filter when the player is in water or lava. Implemented the crosshair at the center of screen.}
            \resumeItem{Two physics modes that can be toggled: normal mode that follows gravity on the ground and buoyancy in water; fly mode that allows free movement in the world across obstacles.}
            \resumeItem{Character-environment interaction: breaking and placing blocks in the world.}
            \resumeItem{Project demo: \url{https://youtu.be/jRb4EHV5KQI}}
          \resumeItemListEnd
          
    \resumeProjectHeading
          {\textbf{TremorChecker} $|$ \emph{Python, Raspberry Pi, Microchip MGC}}{Jan 2021 -- Jun 2023}
          \resumeItemListStart
            \resumeItem{Designed and built a Parkinson’s disease early screening device that calculates human hand’s tremor frequency by creating an electrostatic field to detect the electric potential change.}
            \resumeItem{Implemented the PD screening device with Microchip MGC3030 sensor and Raspberry Pi 4b developing board.}
            \resumeItem{Used Python to write the software that manages the data input and report file output of the device.}
            \resumeItem{Conducted experiments with PD patients and performed screening to 50+ local residents in Shenzhen.}
            \resumeItem{Obtained utility patent in China and participated in the 2023 International Science and Engineering Fair (ISEF).}
          \resumeItemListEnd
          
    \resumeSubHeadingListEnd



\enlargethispage{20pt}


%
%-----------PROGRAMMING SKILLS-----------
\section{Technical Skills}
 \begin{itemize}[leftmargin=0.15in, label={}]
    \small{\item{
     \textbf{Languages}{: C/C++, CUDA, GLSL, eBPF, Java, Python, Assembly, LATEX} \\
     \textbf{Tools}{: Qt, Nvidia nSight, VS Code, Docker, Jupyter Notebook, VMWare, IntelliJ, CloudLab, WebHTML} \\
     \textbf{Knowledge}{: Algorithms, Data Structures, Data Processing, Statistics Methods, Linux Kernel, Embedded Systems, Graphical APIs, Graphical Rendering, Testing Procedures, Network Protocols, OS Design, VM Deployment, Path Tracing, Real-time Ray Tracing, Rasterization, OpenGL Rendering Pipeline, Procedural Asset Generation, Linux Kernel Development, Embedded System Development, Electrical Circuits Design}
    }}
 \end{itemize}


%-------------------------------------------
\end{document}
